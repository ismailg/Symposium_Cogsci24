% Template for Cogsci submission with R Markdown

% Stuff changed from original Markdown PLOS Template
\documentclass[10pt, letterpaper]{article}

\usepackage{cogsci}
\usepackage{pslatex}
\usepackage{float}
\usepackage{caption}

% amsmath package, useful for mathematical formulas
\usepackage{amsmath}

% amssymb package, useful for mathematical symbols
\usepackage{amssymb}

% hyperref package, useful for hyperlinks
\usepackage{hyperref}

% graphicx package, useful for including eps and pdf graphics
% include graphics with the command \includegraphics
\usepackage{graphicx}

% Sweave(-like)
\usepackage{fancyvrb}
\DefineVerbatimEnvironment{Sinput}{Verbatim}{fontshape=sl}
\DefineVerbatimEnvironment{Soutput}{Verbatim}{}
\DefineVerbatimEnvironment{Scode}{Verbatim}{fontshape=sl}
\newenvironment{Schunk}{}{}
\DefineVerbatimEnvironment{Code}{Verbatim}{}
\DefineVerbatimEnvironment{CodeInput}{Verbatim}{fontshape=sl}
\DefineVerbatimEnvironment{CodeOutput}{Verbatim}{}
\newenvironment{CodeChunk}{}{}

% cite package, to clean up citations in the main text. Do not remove.
\usepackage{apacite}

% KM added 1/4/18 to allow control of blind submission
\cogscifinalcopy

\usepackage{color}

% Use doublespacing - comment out for single spacing
%\usepackage{setspace}
%\doublespacing


% % Text layout
% \topmargin 0.0cm
% \oddsidemargin 0.5cm
% \evensidemargin 0.5cm
% \textwidth 16cm
% \textheight 21cm

\title{Computational Social Cognition: Approaches and challenges}


\author{{\large \bf Ismail Guennouni (i.guennouni.17@ucl.ac.uk)} \\ Department of Experimental Psychology, University College London \AND {\large \bf Joseph M Barnby (joseph.barnby@rhul.ac.uk} \\  Department of Psychology Social and Affective Processes, Royal Holloway, University of London \AND {\large \bf Maarten Speekenbrink (m.speekenbrink@ucl.ac.uk)} \\ Department of Experimental Psychology, University College London }

\newlength{\cslhangindent}
\setlength{\cslhangindent}{1.5em}
\newenvironment{CSLReferences}%
  {}%
  {\par}

\begin{document}

\maketitle

\begin{abstract}


\textbf{Keywords:}
Computational Modelling; Social interaction
\end{abstract}

\hypertarget{introduction}{%
\section{Introduction}\label{introduction}}

Traditional cognitive computational models, which are often derived from
individual-oriented experiments, are limited in their ability to capture
the complexity of human cognition in social settings. These models focus
on isolating individual cognitive functions and predicting how they
operate in isolation. However, human behaviour is inherently social and
interactive, and many cognitive processes in this setting are influenced
by factors that emerge specifically through interaction with others.

First, our own behaviour is shaped by the actions and mental state of
others. Yet, the other individuals' motives, emotions, and intentions
remain largely unobservable and thus, unpredictable. This uncertainty
profoundly impacts our decision-making, as our actions are contingent
not only on our own internal states but also on our ability to infer
other's mental state and their responses. Further, our actions depend
not only on our own mental state but also on our beliefs about who we
are interacting with which can be prone to many biases. Dysfunctions in
this inference process has been linked to the emergence of mental health
disorders (Barnby, Dayan, \& Bell, 2023; Luyten, Campbell, Allison, \&
Fonagy, 2020).

Even when we make precise inferences about the other's mental state, we
face the difficulty of keeping an accurate model of the interaction
partner. In social settings, there is a path dependence, in which agents
learn about others and best respond to the history of the interaction.
This can create feedback cycles that profoundly impact and alter the
group's behaviour at a larger scale. These difficulties are compounded
by the fact that social rewards in this context are inherently dynamic
and context-dependent, making them challenging to parametrize in
traditional models (FeldmanHall \& Nassar, 2021). Additionally, when
these interactions occur among a diverse group within a population,
characterized by varying cognitive characteristics or objectives, it
becomes challenging to extend cognitive models from individual to
collective levels (Johnson, 2009).

To address these challenges, there is a growing consensus on the need
for computational models that are not only mathematically tractable but
also psychologically plausible, offering insights that are both
scientifically robust and socially relevant. Such models should not only
encapsulate the complexities of individual cognitive processes but also
the intricacies of social interactions and relationships: This involves
constructing models that can handle the unpredictable nature of social
interactions, the contextual variability of social rewards, and the
interdependencies that characterize human relationships (FeldmanHall \&
Nassar, 2021).

This symposium aims to bring together researchers and thinkers in the
field to explore these challenges and opportunities. Through a
combination of theoretical insights and empirical findings, we aim to
chart a course for future research that can more accurately model and
understand the complexities of social interaction cognition. By bridging
the gap between computational models and the richness of human social
interactions, we hope to contribute to a better understanding of how we
navigate, interpret, and engage with our social world.

\hypertarget{contributors}{%
\section{Contributors}\label{contributors}}

\textbf{Joseph Barnby} is a computational and cognitive neuroscientist,
and Assistant Professor at Royal Holloway, University of London. He
received his BSc at the University of Leicester in Psychology and
Neuroscience, his MSc in Clinical Mental Health from UCL, and his PhD in
Cognitive Neuroscience from KCL. His lab -- the Social Computation and
Representation Lab -- is interested in the brain basis of social
interaction, using and developing computational models imbued with
Theory of Mind to provide novel theoretical frameworks and empirical
evidence to inform the aetiology of psychiatric disorder, and to develop
more dynamic artificial systems.

\textbf{Ismail Guennouni} is a computational cognitive scientist and
Postdoctoral Research Fellow at the AI Health Innovation Cluster
associated with the University of Heidelberg and the ZI Mannheim. He
received his MSc in Cognitive Science and his PhD in Psychology from UCL
with a focus on experimental and computations approaches to strategic
social interaction. His research investigates the differential aspects
of social learning in mental health disorders. He is interested in
exploring how cognitive interventions, combined with computational
modelling of behaviour can help address social learning dysfunction
inherent in many mental health disorders.

\textbf{Julian Jara-Ettinger} is an associate professor of psychology at
Yale University, with affiliations to the Computer Science department,
the Cognitive Science program, and the Wu Tsai institute. Julian
received his bachelor's degree in physics and mathematics at the
Universidad Michoacana in Mexico and his PhD in Cognitive Science at
MIT. At Yale, Julian's research group--the computational social
cognition lab--aims to characterize the representations and computations
that support human social cognition, understand how they emerge and
develop, and use them to build more human-like machine social
intelligence.

\textbf{Maarten Speekenbrink} is Professor of Mathematical Psychology at
UCL. His research combines computational models and behavioural
experiments to identify core elements of human learning and decision
making. In recent work, he focuses on how these processes operate in
social interactions with other agents.

\hypertarget{between-prudence-and-paranoia-the-neural-and-computational-basis-of-strategic-mentalising-gone-right-and-wrong}{%
\section{Between prudence and paranoia: the neural and computational
basis of strategic mentalising gone right and
wrong}\label{between-prudence-and-paranoia-the-neural-and-computational-basis-of-strategic-mentalising-gone-right-and-wrong}}

\begin{center}
Joseph Barnby
\end{center}

Strategic reasoning is essential to avoid deception. Too much vigilance
can however lead to false beliefs about a partner's intended harm. This
talk will focus on recent work developing mathematical models of how
humans build recursive maps of their social partners for strategic
interaction, testing which neurochemical and social factors cause this
ability to go awry, and how this may explain psychopathological
symptoms. Making small mis-calibrated changes to the way in which
artificial agents interact causes social interaction to break down and
can account for a several psychopathological symptoms observed in the
clinic. It also identifies the necessity of calibrating artificial
systems to their users to ensure ingenuous behaviour is not mistaken as
threat.

\hypertarget{hidden-markov-models-to-capture-the-dynamics-of-strategic-interaction-and-build-adaptive-human-like-agents}{%
\section{Hidden Markov models to capture the dynamics of strategic
interaction and build adaptive human-like
agents}\label{hidden-markov-models-to-capture-the-dynamics-of-strategic-interaction-and-build-adaptive-human-like-agents}}

\begin{center}
Ismail Guennouni
\end{center}

In social environments, the outcomes of our actions are influenced by
the behavior of others, necessitating mutual adaptation. Static opponent
models are inadequate when dealing with players who are also learning
and modelling our strategies. In this talk, I will introduce a framework
based on hidden Markov models, tailored to capture the evolving dynamics
of interactions in repeated economic games. I will show how this
approach effectively characterizes player behavior through analysis of
dyadic human exchange data. I will then explore how these HMM models
facilitate the development of adaptive artificial agents that can
emulate human behavior in economic games, whilst offering a higher
degree of experimental control. Additionally, I will demonstrate the
application of these agents in intervention studies aimed at bolstering
cooperative behavior in such games.

\hypertarget{social-representations-as-probabilistic-programs}{%
\section{Social representations as probabilistic
programs}\label{social-representations-as-probabilistic-programs}}

\begin{center}
Julian Jara-Ettinger
\end{center}

Virtually all areas of uniquely-human intelligence, from moral reasoning
to language understanding, rely on social cognition. Characterizing its
computational structure is therefore a central challenge in cognitive
science and critical towards engineering more human-like AI. In this
talk I will present a computational framework of social cognition, where
agent behavior is represented hierarchically through a combination of
symbolic propositional programs, with internal non-symbolic continuous
representations of mental state contents. I show experimental evidence
that this framework captures how people make sense of behavior,
including cases with complex reward structures that are not adequately
captured by previous approaches. Finally, I will also discuss some
ethical questions that arise when considering the role of this
technology in potential surveillance applications.

\hypertarget{panel-discussion}{%
\section{Panel discussion}\label{panel-discussion}}

\begin{center}
Maarten Speekenbrink
\end{center}

The talks will be followed by a panel discussion, introduced by Maarten
Speekenbrink. He will aim to integrate insights from the four talks, and
highlight future challenges and directions for the field. These will
then be discussed by the panel of speakers.

\hypertarget{references}{%
\section{References}\label{references}}

\setlength{\parindent}{-0.1in} 
\setlength{\leftskip}{0.125in}

\noindent

\hypertarget{refs}{}
\begin{CSLReferences}{1}{0}
\leavevmode\vadjust pre{\hypertarget{ref-barnby2023formalising}{}}%
Barnby, J. M., Dayan, P., \& Bell, V. (2023). Formalising social
representation to explain psychiatric symptoms. \emph{Trends in
Cognitive Sciences}.

\leavevmode\vadjust pre{\hypertarget{ref-feldmanhall2021computational}{}}%
FeldmanHall, O., \& Nassar, M. R. (2021). The computational challenge of
social learning. \emph{Trends in Cognitive Sciences}, \emph{25}(12),
1045--1057.

\leavevmode\vadjust pre{\hypertarget{ref-johnson2009simply}{}}%
Johnson, N. (2009). \emph{Simply complexity: A clear guide to complexity
theory}. Simon; Schuster.

\leavevmode\vadjust pre{\hypertarget{ref-luyten2020mentalizing}{}}%
Luyten, P., Campbell, C., Allison, E., \& Fonagy, P. (2020). The
mentalizing approach to psychopathology: State of the art and future
directions. \emph{Annual Review of Clinical Psychology}, \emph{16},
297--325.

\end{CSLReferences}

\bibliographystyle{apacite}


\end{document}
