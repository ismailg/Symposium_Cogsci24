% Template for Cogsci submission with R Markdown

% Stuff changed from original Markdown PLOS Template
\documentclass[10pt, letterpaper]{article}

\usepackage{cogsci}
\usepackage{pslatex}
\usepackage{float}
\usepackage{caption}

% amsmath package, useful for mathematical formulas
\usepackage{amsmath}

% amssymb package, useful for mathematical symbols
\usepackage{amssymb}

% hyperref package, useful for hyperlinks
\usepackage{hyperref}

% graphicx package, useful for including eps and pdf graphics
% include graphics with the command \includegraphics
\usepackage{graphicx}

% Sweave(-like)
\usepackage{fancyvrb}
\DefineVerbatimEnvironment{Sinput}{Verbatim}{fontshape=sl}
\DefineVerbatimEnvironment{Soutput}{Verbatim}{}
\DefineVerbatimEnvironment{Scode}{Verbatim}{fontshape=sl}
\newenvironment{Schunk}{}{}
\DefineVerbatimEnvironment{Code}{Verbatim}{}
\DefineVerbatimEnvironment{CodeInput}{Verbatim}{fontshape=sl}
\DefineVerbatimEnvironment{CodeOutput}{Verbatim}{}
\newenvironment{CodeChunk}{}{}

% cite package, to clean up citations in the main text. Do not remove.
\usepackage{apacite}

% KM added 1/4/18 to allow control of blind submission
\cogscifinalcopy

\usepackage{color}

% Use doublespacing - comment out for single spacing
%\usepackage{setspace}
%\doublespacing


% % Text layout
% \topmargin 0.0cm
% \oddsidemargin 0.5cm
% \evensidemargin 0.5cm
% \textwidth 16cm
% \textheight 21cm

\title{Computational Approaches to Cognition in Social Interaction}


\author{{\large \bf Maarten Speekenbrink (m.speekenbrink@ucl.ac.uk)} \\ Department of Experimental Psychology, University College London  \AND {\large \bf Joseph M Barnby (Joseph.Barnby@rhul.ac.uk} \\  Department of Psychology Social and Affective Processes, Royal Holloway, University of London \AND {\large \bf Ismail Guennouni (i.guennouni.17@ucl.ac.uk)} \\ Department of Experimental Psychology, University College London }

\newlength{\cslhangindent}
\setlength{\cslhangindent}{1.5em}
\newenvironment{CSLReferences}%
  {}%
  {\par}

\begin{document}

\maketitle

\begin{abstract}


\textbf{Keywords:}
Computational Modelling; Social interaction
\end{abstract}

\hypertarget{introduction}{%
\section{Introduction}\label{introduction}}

Traditional cognitive computational models, which are often derived from
individual-oriented experiments, are limited in their ability to capture
the complexity of human cognition in social settings. These models focus
on isolating individual cognitive functions and predicting how they
operate in isolation. However, human behaviour is inherently social and
interactive, and many cognitive processes in this setting are influenced
by factors that emerge specifically through interaction with others.

First, our own behaviour is shaped by the actions and mental state of
others. Yet, the other individuals' motives, emotions, and intentions
remain largely unobservable and thus, unpredictable. This uncertainty
profoundly impacts our decision-making, as our actions are contingent
not only on our own internal states but also on our ability to infer
other's mental state and their responses. Further, our actions depend
not only on our own mental state but also on our beliefs about who we
are interacting with which can be prone to many biases. Dysfunctions in
this inference process has been linked to the emergence of mental health
disorders (Barnby, Dayan, \& Bell, 2023; Luyten, Campbell, Allison, \&
Fonagy, 2020).

Even when we make precise inferences about the other's mental state, we
face the difficulty of keeping an accurate model of the interaction
partner. In social settings, there is a path dependence, in which agents
learn about others and best respond to the history of the interaction.
This can create feedback cycles that profoundly impact and alter the
group's behaviour at a larger scale. These difficulties are compounded
by the fact that social rewards in this context are inherently dynamic
and context-dependent, making them challenging to parametrize in
traditional models (FeldmanHall \& Nassar, 2021). Additionally, when
these interactions occur among a diverse group within a population,
characterized by varying cognitive characteristics or objectives, it
becomes challenging to extend cognitive models from individual to
collective levels (Johnson, 2009).

To address these challenges, there is a growing consensus on the need
for computational models that are not only mathematically tractable but
also psychologically plausible, offering insights that are both
scientifically robust and socially relevant. Such models should not only
encapsulate the complexities of individual cognitive processes but also
the intricacies of social interactions and relationships: This involves
constructing models that can handle the unpredictable nature of social
interactions, the contextual variability of social rewards, and the
interdependencies that characterize human relationships (FeldmanHall \&
Nassar, 2021).

This symposium aims to bring together researchers and thinkers in the
field to explore these challenges and opportunities. Through a
combination of theoretical insights and empirical findings, we aim to
chart a course for future research that can more accurately model and
understand the complexities of social interaction cognition. By bridging
the gap between computational models and the richness of human social
interactions, we hope to contribute to a better understanding of how we
navigate, interpret, and engage with our social world.

\hypertarget{speaker-biographies-talks}{%
\section{Speaker Biographies \& Talks}\label{speaker-biographies-talks}}

\hypertarget{references}{%
\section{References}\label{references}}

\setlength{\parindent}{-0.1in} 
\setlength{\leftskip}{0.125in}

\noindent

\hypertarget{refs}{}
\begin{CSLReferences}{1}{0}
\leavevmode\vadjust pre{\hypertarget{ref-barnby2023formalising}{}}%
Barnby, J. M., Dayan, P., \& Bell, V. (2023). Formalising social
representation to explain psychiatric symptoms. \emph{Trends in
Cognitive Sciences}.

\leavevmode\vadjust pre{\hypertarget{ref-feldmanhall2021computational}{}}%
FeldmanHall, O., \& Nassar, M. R. (2021). The computational challenge of
social learning. \emph{Trends in Cognitive Sciences}, \emph{25}(12),
1045--1057.

\leavevmode\vadjust pre{\hypertarget{ref-johnson2009simply}{}}%
Johnson, N. (2009). \emph{Simply complexity: A clear guide to complexity
theory}. Simon; Schuster.

\leavevmode\vadjust pre{\hypertarget{ref-luyten2020mentalizing}{}}%
Luyten, P., Campbell, C., Allison, E., \& Fonagy, P. (2020). The
mentalizing approach to psychopathology: State of the art and future
directions. \emph{Annual Review of Clinical Psychology}, \emph{16},
297--325.

\end{CSLReferences}

\bibliographystyle{apacite}


\end{document}
