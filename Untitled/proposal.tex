% Template for Cogsci submission with R Markdown

% Stuff changed from original Markdown PLOS Template
\documentclass[10pt, letterpaper]{article}

\usepackage{cogsci}
\usepackage{pslatex}
\usepackage{float}
\usepackage{caption}

% amsmath package, useful for mathematical formulas
\usepackage{amsmath}

% amssymb package, useful for mathematical symbols
\usepackage{amssymb}

% hyperref package, useful for hyperlinks
\usepackage{hyperref}

% graphicx package, useful for including eps and pdf graphics
% include graphics with the command \includegraphics
\usepackage{graphicx}

% Sweave(-like)
\usepackage{fancyvrb}
\DefineVerbatimEnvironment{Sinput}{Verbatim}{fontshape=sl}
\DefineVerbatimEnvironment{Soutput}{Verbatim}{}
\DefineVerbatimEnvironment{Scode}{Verbatim}{fontshape=sl}
\newenvironment{Schunk}{}{}
\DefineVerbatimEnvironment{Code}{Verbatim}{}
\DefineVerbatimEnvironment{CodeInput}{Verbatim}{fontshape=sl}
\DefineVerbatimEnvironment{CodeOutput}{Verbatim}{}
\newenvironment{CodeChunk}{}{}

% cite package, to clean up citations in the main text. Do not remove.
\usepackage{apacite}

% KM added 1/4/18 to allow control of blind submission
\cogscifinalcopy

\usepackage{color}

% Use doublespacing - comment out for single spacing
%\usepackage{setspace}
%\doublespacing


% % Text layout
% \topmargin 0.0cm
% \oddsidemargin 0.5cm
% \evensidemargin 0.5cm
% \textwidth 16cm
% \textheight 21cm

\title{Computational Social Cognition: Approaches and challenges}


\author{{\large \bf Ismail Guennouni (ismail.guennouni@zi-mannheim.de)} Z.I Mannheim and University of Heidelberg \AND {\large \bf Joseph M Barnby (joseph.barnby@rhul.ac.uk)} Royal Holloway, University of London \AND {\large \bf Julian Jara-Ettinger (julian.jara-ettinger@yale.edu)} Yale University \AND {\large \bf Rebecca Saxe (saxe@mit.edu)} MIT \AND {\large \bf Maarten Speekenbrink (m.speekenbrink@ucl.ac.uk)} University College London}

\newlength{\cslhangindent}
\setlength{\cslhangindent}{1.5em}
%\newenvironment{CSLReferences}%
%  {}%
%  {\par}

\newlength{\csllabelwidth}
\setlength{\csllabelwidth}{3em}
\newlength{\cslentryspacingunit} % times entry-spacing
\setlength{\cslentryspacingunit}{\parskip}

% for Pandoc 2.8 to 2.10.1
\newenvironment{cslreferences}%
{}%
{\par}
% For Pandoc 2.11+
\newenvironment{CSLReferences}[2] % #1 hanging-ident, #2 entry spacing
{% don't indent paragraphs
	\setlength{\parindent}{0pt}
	% turn on hanging indent if param 1 is 1
	\ifodd #1
	\let\oldpar\par
	\def\par{\hangindent=\cslhangindent\oldpar}
	\fi
	% set entry spacing
	%\setlength{\parskip}{#2\cslentryspacingunit}
}%
{}
\usepackage{calc}
\newcommand{\CSLBlock}[1]{#1\hfill\break}
\newcommand{\CSLLeftMargin}[1]{\parbox[t]{\csllabelwidth}{#1}}
\newcommand{\CSLRightInline}[1]{\parbox[t]{\linewidth - \csllabelwidth}{#1}\break}
\newcommand{\CSLIndent}[1]{\hspace{\cslhangindent}#1}


\begin{document}

\maketitle

% \begin{abstract}
% 
% 
% \textbf{Keywords:}
% Computational Models; Social interaction; Theory of Mind
% \end{abstract}

\hypertarget{introduction}{%
\section{Introduction}\label{introduction}}

Predicting the actions and reactions of others is crucial to successful
social interaction. When deciding whether to bluff in a game of poker,
we consider the chances that the other players will fold or continue to
play and unmask our bluff. When deciding whether to tell our boss that
their plans are likely to have adverse effects, we consider a range of
reactions, from being grateful for our honesty to being dismissed out of
spite. Such predictions are highly uncertain and complex, not least
because the other's (re)actions usually result from them making equally
complex and uncertain inferences about us. Nevertheless, we are often
remarkably successful -- although sometimes utterly wrong -- in our
social inferences. How do we explain these successes and failures?

Theory of Mind (ToM) or mentalizing, referring to the ability to
represent another's latent mental states (e.g.~thoughts, beliefs,
motivations, and emotions) is a central concept in theories of social
inference (Baker, Jara-Ettinger, Saxe, \& Tenenbaum, 2017; Barnby,
Bellucci, et al., 2023). That predictions of another's actions derive
from inferences of such latent states is a vague --yet meaningful --
statement. How are these inferences made? And what are the benefits of
representing others in terms of their beliefs, motivations, and
emotions, over and above directly predicting their (re)actions?

There is a growing consensus that computational models are required to
reduce this ambiguity and improve the falsifiability of theories of
social interaction (FeldmanHall \& Nassar, 2021). These models should be
both mathematically tractable and psychologically plausible, offering
insights that are scientifically robust and socially relevant. Given the
complexities of social interactions and inference, it is unlikely that
models which work well in non-social domains (e.g.~standard
reinforcement learning models) can be called upon without modification.

This symposium brings together researchers who have taken up the
challenge of computational social cognition with a variety of
approaches, including generative Bayesian models (Barnby, Dayan, \&
Bell, 2023), inverse reinforcement learning (Jara-Ettinger, 2019),
unsupervised clustering techniques (Guennouni \& Speekenbrink, 2022),
and predictive coding (Koster-Hale \& Saxe, 2013). The aim of this
symposium is to identify the unique challenges that face computational
social cognition, how these are met by the different approaches, and
what remains to be addressed in the future.

\hypertarget{contributors}{%
\section{Contributors}\label{contributors}}

\textbf{Joseph Barnby} is a computational and cognitive neuroscientist,
and Assistant Professor at Royal Holloway, University of London. He
received his PhD in Cognitive Neuroscience from King's College London.
His lab -- the Social Computation and Representation Lab -- is
interested in the brain basis of social interaction, using and
developing computational models imbued with Theory of Mind to provide
novel theoretical frameworks and empirical evidence to inform the
aetiology of psychiatric disorder, and to develop more dynamic
artificial systems.

\textbf{Ismail Guennouni} is a computational cognitive scientist and
Postdoctoral Research Fellow at the AI Health Innovation Cluster He
received his PhD in Psychology from UCL with a focus on experimental and
computations approaches to strategic social interaction. His research
investigates the differential aspects of social learning in mental
health disorders. He is interested in exploring how cognitive
interventions, combined with computational modelling of behaviour can
help address social learning dysfunction inherent in many mental health
disorders.

\textbf{Julian Jara-Ettinger} is an associate professor of psychology at
Yale University, with affiliations to the Computer Science department,
the Cognitive Science program, and the Wu Tsai institute. Julian
received his PhD in Cognitive Science at MIT. At Yale, Julian's research
group--the computational social cognition lab--aims to characterize the
representations and computations that support human social cognition,
understand how they emerge and develop, and use them to build more
human-like machine social intelligence.

\textbf{Rebecca Saxe} is a social cognitive neuroscientist, John W Jarve
(1978) Professor of Cognitive Neuroscience, and Associate Dean of
Science at MIT. She received her her PhD in Cognitive Science at MIT,
and was a junior fellow in the Harvard Society of Fellows. Her lab
studies the development and neural basis of human cognition, focusing on
social cognition.

\textbf{Maarten Speekenbrink} is Professor of Mathematical Psychology at
UCL. His research combines computational models and behavioural
experiments to identify core elements of human learning and decision
making. In recent work, he focuses on how these processes operate in
social interactions with other agents.

\hypertarget{between-prudence-and-paranoia-the-neural-and-computational-basis-of-strategic-mentalising-gone-right-and-wrong}{%
\section{Between prudence and paranoia: the neural and computational
basis of strategic mentalising gone right and
wrong}\label{between-prudence-and-paranoia-the-neural-and-computational-basis-of-strategic-mentalising-gone-right-and-wrong}}

\begin{center}
Joseph Barnby
\end{center}

Strategic reasoning is essential to avoid deception. Too much vigilance
can however lead to false beliefs about a partner's intended harm. This
talk will focus on recent work developing mathematical models of how
humans build recursive maps of their social partners for strategic
interaction, testing which neurochemical and social factors cause this
ability to go awry, and how this may explain psychopathological
symptoms. Making small mis-calibrated changes to the way in which
artificial agents interact causes social interaction to break down and
can account for a several psychopathological symptoms observed in the
clinic. It also identifies the necessity of calibrating artificial
systems to their users to ensure ingenuous behaviour is not mistaken as
threat.

\hypertarget{hidden-markov-models-to-capture-the-dynamics-of-strategic-interaction-and-build-adaptive-human-like-agents}{%
\section{Hidden Markov models to capture the dynamics of strategic
interaction and build adaptive human-like
agents}\label{hidden-markov-models-to-capture-the-dynamics-of-strategic-interaction-and-build-adaptive-human-like-agents}}

\begin{center}
Ismail Guennouni
\end{center}

In social environments, the outcomes of our actions are influenced by
the behavior of others, necessitating mutual adaptation. Static opponent
models are inadequate when dealing with players who are also learning
and modelling our strategies. In this talk, I will introduce a framework
based on hidden Markov models, tailored to capture the evolving dynamics
of interactions in repeated economic games. I will show how this
approach effectively characterizes player behavior through analysis of
dyadic human exchange data. I will then explore how these HMM models
facilitate the development of adaptive artificial agents that can
emulate human behavior in economic games, whilst offering a higher
degree of experimental control. Additionally, I will demonstrate the
application of these agents in intervention studies aimed at bolstering
cooperative behavior in such games.

\hypertarget{social-representations-as-probabilistic-programs}{%
\section{Social representations as probabilistic
programs}\label{social-representations-as-probabilistic-programs}}

\begin{center}
Julian Jara-Ettinger
\end{center}

Virtually all areas of uniquely-human intelligence, from moral reasoning
to language understanding, rely on social cognition. Characterizing its
computational structure is therefore a central challenge in cognitive
science and critical towards engineering more human-like AI. In this
talk I will present a computational framework of social cognition, where
agent behavior is represented hierarchically through a combination of
symbolic propositional programs, with internal non-symbolic continuous
representations of mental state contents. I show experimental evidence
that this framework captures how people make sense of behavior,
including cases with complex reward structures that are not adequately
captured by previous approaches. Finally, I will also discuss some
ethical questions that arise when considering the role of this
technology in potential surveillance applications.

\hypertarget{using-inverse-planning-to-learn-from-and-communicate-with-social-actions}{%
\section{Using inverse planning to learn from and communicate with
social
actions}\label{using-inverse-planning-to-learn-from-and-communicate-with-social-actions}}

\begin{center}
Rebecca Saxe
\end{center}

Any social action has multiple possible explanations. For example,
consider punishment. If an authority chooses to punish a norm violation,
one possible explanation is that the norm violation is morally wrong,
and the authority is impartial. Another possible explanation is that the
norm violation is relatively innocuous, and the authority is biased
against the target. I will present a formal cognitive model of how
people accommodate the ambiguity of social actions, and rationally
jointly update beliefs about the situation (i.e.~the norm violation) and
the actor (i.e.~the authority's motives), depending on their priors.
This model predicts when the beliefs of different people observing the
same social interactions will rationally diverge, and fits human
inferences across three studies (N=1260). The model of observers'
inferences can be further embedded, recursively, in a model of planning,
to explain how people anticipate the reputational consequences of their
decisions and plan communicative social actions.

\hypertarget{panel-discussion}{%
\section{Panel discussion}\label{panel-discussion}}

\begin{center}
Maarten Speekenbrink
\end{center}

The talks will be followed by a panel discussion, introduced by Maarten
Speekenbrink. He will aim to integrate insights from the four talks, and
highlight future challenges and directions for the field. These will
then be discussed by the panel of speakers.

\hypertarget{references}{%
\section{References}\label{references}}

\setlength{\parindent}{-0.1in} 
\setlength{\leftskip}{0.125in}

\noindent

\small

\hypertarget{refs}{}
\begin{CSLReferences}{1}{0}
\leavevmode\vadjust pre{\hypertarget{ref-baker2017rational}{}}%
Baker, C. L., Jara-Ettinger, J., Saxe, R., \& Tenenbaum, J. B. (2017).
Rational quantitative attribution of beliefs, desires and percepts in
human mentalizing. \emph{Nature Human Behaviour}, \emph{1}(4), 0064.

\leavevmode\vadjust pre{\hypertarget{ref-barnby2023beyond}{}}%
Barnby, J. M., Bellucci, G., Alon, N., Schilbach, L., Bell, V., Frith,
C., \& Dayan, P. (2023). Beyond theory of mind: A formal framework for
social inference and representation. \emph{PsyArXiv}.
http://doi.org/\href{https://doi.org/10.31234/osf.io/cmgu7}{10.31234/osf.io/cmgu7}

\leavevmode\vadjust pre{\hypertarget{ref-barnby2023formalising}{}}%
Barnby, J. M., Dayan, P., \& Bell, V. (2023). Formalising social
representation to explain psychiatric symptoms. \emph{Trends in
Cognitive Sciences}, \emph{27}, 317--332.

\leavevmode\vadjust pre{\hypertarget{ref-feldmanhall2021computational}{}}%
FeldmanHall, O., \& Nassar, M. R. (2021). The computational challenge of
social learning. \emph{Trends in Cognitive Sciences}, \emph{25}(12),
1045--1057.

\leavevmode\vadjust pre{\hypertarget{ref-guennouni2022transfer}{}}%
Guennouni, I., \& Speekenbrink, M. (2022). Transfer of learned opponent
models in zero sum games. \emph{Computational Brain \& Behavior},
\emph{5}(3), 326--342.

\leavevmode\vadjust pre{\hypertarget{ref-jara2019theory}{}}%
Jara-Ettinger, J. (2019). Theory of mind as inverse reinforcement
learning. \emph{Current Opinion in Behavioral Sciences}, \emph{29},
105--110.

\leavevmode\vadjust pre{\hypertarget{ref-koster2013theory}{}}%
Koster-Hale, J., \& Saxe, R. (2013). Theory of mind: A neural prediction
problem. \emph{Neuron}, \emph{79}(5), 836--848.

\end{CSLReferences}

\bibliographystyle{apacite}


\end{document}
